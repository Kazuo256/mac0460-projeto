\documentclass[a4paper,11pt]{article}
\usepackage[T1]{fontenc}
\usepackage[brazilian]{babel}
\usepackage[utf8]{inputenc}
\usepackage{lmodern}
\usepackage{multirow}

\title{MAC0460 - 2º semestre de 2012 \\ Relatório do Projeto}
\author{
  Wilson Kazuo Mizutani\\
  Número USP: 6797230
  \and
  Gustavo Teixeira da Cunha Coelho\\
  Número USP: 6797334
}

\begin{document}
\maketitle

\section{Decisões de projeto}

\begin{itemize}
  \item Classificadores usados:
  \begin{itemize}
    \item[-] Normal Bayes
    \item[-] K-Nearest Neighbours
    \item[-] Decisions Tree
    \item[-] Random Trees
  \end{itemize}
  
  \item Detectores e Extratores
    \begin{itemize}
      \item Detectores usados:
        \begin{itemize}
          \item STAR
          \item SURF
          \item SIFT
          \item ORB*
        \end{itemize}

      \item Adaptadores:
        \begin{itemize} 
          \item Pyramid (usamos cada detector com e sem esse adaptador)
        \end{itemize}

      \item Extratores usados:
        \begin{itemize} 
          \item SURF
          \item SIFT
        \end{itemize}

    \end{itemize}

\end{itemize}

Motivos de uso:
\begin{itemize}
  \item Usamos os que não deram problema de execução.
  \item Os problemas de execução eram ou incompatibilidade de formatos das matrizes, ou falhas de segmentação dentro do OpenCV.
\end{itemize}

* A combinação de detector/extrator ORB+SIFT causava falha de segmentação.

\section{Otimizações utilizadas}
Usamos o recurso de salvar classificadores em arquivos sempre que a respectiva implementação permitia.\\*
Também salvamos em arquivos os vocabulários gerados para cada combinação Detector-Extractor. Eles eram o maior gargalo das execuções (alguns passavam de 20 minutos). Como esses arquivos podiam ser aproveitados de um classificador para outro, ganhamos bastante tempo na hora de rodar as baterias de testes.\\*
Graças a isso, a parte relevante das medidas de tempo se reduziram ao tempo de detecção de KeyPoints/descritores e ao de extração de histogramas, que só dependem da combinação Detector/Extrator e não do classificador usado.

\section{Resultados obtidos}

Abaixo segue uma tabela contendo o erro máximo de um classificador dado os resultados dos testes. Esse erro é calculado a partir da fórmula:\\* $ Erro_T(g)+Z_n * \sqrt{\frac{Erro_T(g)(1 - Erro_T(g))}{n}} $\\*
Onde n é o número de amostras de teste, $Erro_T$ é a porcentagem de erro do classificador e $Z_n$ é uma constante indicando o intervalo de confiança. Neste caso usamos 90\% de intervalo de confiança, mas como tal intervalo denota o intervalo entre o erro máximo e o erro mínimo, consideramos apenas o erro máximo, que então é tido com 95\% de confiança.\\*

\begin{tabular}{|c|c|c|c|c|}
    \hline
     & Bayes & kNN & Random Trees & Decision Tree \\
    \hline
        STAR-SURF & 48,38\% & 77,35\% & 72,52\% & 83,57\% \\
        SURF-SURF & 23,31\% & 46,56\% & 35,29\% & 72,52\% \\
        SIFT-SURF & 53,77\% & 65,88\% & 53,77\% & 78,93\%\\
        ORB-SURF & 25,38\% & 48,38\% & 46,56\% & 70,88\% \\
        PyramidSTAR-SURF & 50,19\% & 75,75\% & 62,48\% & 77,35\% \\
        PyramidSURF-SURF & 29,41\% & 42,87\% & 42,87\% & 80,49\% \\
        PyramidSIFT-SURF & 42,87\% & 57,29\% & 46,56\% & 60,77\% \\
        PyramidORB-SURF & 19,08\% & 31,39\% & 29,41\% & 65,88\% \\
        STAR-SIFT & 21,22\% & 27,41\% & 27,41\% & 55,54\% \\
        SURF-SIFT & 27,41\% & 27,41\% & 31,39\% & 55,54\% \\
        SIFT-SIFT & 31,39\% & 50,19\% & 44,72\% & 67,56\% \\
        PyramidSTAR-SIFT & 12,35\% & 14,66\% & 23,31\% & 46,56\% \\
        PyramidSURF-SIFT & 27,41\% & 31,39\% & 27,41\% & 46,56\% \\
        PyramidSIFT-SIFT & 12,35\% & 39,11\% & 35,29\% & 64,19\% \\
        PyramidORB-SIFT & 23,31\% & 29,41\% & 21,29\% & 44,72\% \\
        
    \hline
\end{tabular}
\bigskip

É fácil deduzir da tabela que o classificador Bayesiano é melhor que os outros 3. De fato, ele foi melhor em todas as combinações de Detector-Extrator, exceto contra o Classificador Random Trees usando a combinação PyramidORB-SIFT.\\*

Abaixo segue uma tabela do tempo tomado para detecção de KeyPoints, extração de descritores e extração de histogramas. O tempo para criar o vocabulário e treino do classificador é ignorado aqui pois, como já mencionado, ambos o classificador e o vocabulário podem ser salvos em arquivos para uso posterior.\\*

\begin{tabular}{|c|c|c|}
    \hline
    Feature &  \multicolumn{2}{|c|}{Descriptor Extractor} \\
    \cline{2-3}
     Detectors & SURF & SIFT \\
    \hline
    STAR & 11,25 & 57 \\
    SURF & 108,5 & 524,5 \\
    SIFT  & 26 & 47,25 \\
    ORB  & 50 & X \\
    PyramidSTAR & 13,75 & 66,5 \\
    PyramidSURF & 172,75 & 1396 \\
    PyramidSIFT & 37 & 111,25 \\
    PyramidORB & 222,5 & 1634,75 \\
    \hline
  \end{tabular}
  \bigskip

Abaixo seguem as matrizes de confusão de cada um dos melhores resultados de Descritor/Extrator para cada classificador.\\*

  \hspace{-75pt}
  \begin{tabular}{|cc|c|c|c|c|c|c|c|}
    \hline
    \multicolumn{2}{|c|}{Classificador Normal Bayes} &
    \multicolumn{6}{|c|}{Respostas do Classificador} \\
    \cline{3-8}
    \multicolumn{2}{|c|}{com PyramidSIFT + SIFT}
    & Cristo & Muralha & Panteão & Pirâmides & Taj Mahal & Torre Eiffel \\
    \hline
    \multirow{6}{*}{Classes Reais}
    & \multicolumn{1}{|c|}{Cristo} & 11 & 0 & 0 & 0 & 0 & 0 \\
    \cline{2-8}
    & \multicolumn{1}{|c|}{Muralha} & 1 & 8 & 0 & 0 & 1 & 0\\
    \cline{2-8}
    & \multicolumn{1}{|c|}{Panteão} & 0 & 0 & 8 & 0 & 0 & 0\\
    \cline{2-8}
    & \multicolumn{1}{|c|}{Pirâmides} & 0 & 1 & 0 & 8 & 1 & 0\\
    \cline{2-8}
    & \multicolumn{1}{|c|}{Taj Mahal} & 0 & 0 & 0 & 0 & 9 & 0\\
    \cline{2-8}
    & \multicolumn{1}{|c|}{Torre Eiffel} & 0 & 0 & 0 & 0 & 0 & 10\\
    \hline
  \end{tabular}
  \bigskip
  

  \hspace{-75pt}
  \begin{tabular}{|cc|c|c|c|c|c|c|c|}
    \hline
    \multicolumn{2}{|c|}{Classificador kNN} &
    \multicolumn{6}{|c|}{Respostas do Classificador} \\
    \cline{3-8}
    \multicolumn{2}{|c|}{com PyramidSTAR + SIFT}
    & Cristo & Muralha & Panteão & Pirâmides & Taj Mahal & Torre Eiffel \\
    \hline
    \multirow{6}{*}{Classes Reais}
    & \multicolumn{1}{|c|}{Cristo} & 11 & 0 & 0 & 0 & 0 & 0 \\
    \cline{2-8}
    & \multicolumn{1}{|c|}{Muralha} & 2 & 8 & 0 & 0 & 0 & 0\\
    \cline{2-8}
    & \multicolumn{1}{|c|}{Panteão} & 0 & 0 & 8 & 0 & 0 & 0\\
    \cline{2-8}
    & \multicolumn{1}{|c|}{Pirâmides} & 0 & 2 & 0 & 7 & 1 & 0\\
    \cline{2-8}
    & \multicolumn{1}{|c|}{Taj Mahal} & 0 & 0 & 0 & 0 & 9 & 0\\
    \cline{2-8}
    & \multicolumn{1}{|c|}{Torre Eiffel} & 0 & 0 & 0 & 0 & 0 & 10\\
    \hline
  \end{tabular}
  \bigskip


  \hspace{-75pt}
  \begin{tabular}{|cc|c|c|c|c|c|c|c|}
    \hline
    \multicolumn{2}{|c|}{Classificador Random Trees} &
    \multicolumn{6}{|c|}{Respostas do Classificador} \\
    \cline{3-8}
    \multicolumn{2}{|c|}{com PyramidORB + SIFT}
    & Cristo & Muralha & Panteão & Pirâmides & Taj Mahal & Torre Eiffel \\
    \hline
    \multirow{6}{*}{Classes Reais}
    & \multicolumn{1}{|c|}{Cristo} & 10 & 1 & 0 & 0 & 0 & 0 \\
    \cline{2-8}
    & \multicolumn{1}{|c|}{Muralha} & 2 & 8 & 0 & 0 & 0 & 0\\
    \cline{2-8}
    & \multicolumn{1}{|c|}{Panteão} & 0 & 0 & 8 & 0 & 0 & 0\\
    \cline{2-8}
    & \multicolumn{1}{|c|}{Pirâmides} & 2 & 2 & 0 & 6 & 0 & 0\\
    \cline{2-8}
    & \multicolumn{1}{|c|}{Taj Mahal} & 1 & 0 & 0 & 0 & 8 & 0\\
    \cline{2-8}
    & \multicolumn{1}{|c|}{Torre Eiffel} & 0 & 0 & 0 & 0 & 0 & 10\\
    \hline
  \end{tabular}
  \bigskip


\hspace{-75pt}
  \begin{tabular}{|cc|c|c|c|c|c|c|c|}
    \hline
    \multicolumn{2}{|c|}{Classificador Decision Tree} &
    \multicolumn{6}{|c|}{Respostas do Classificador} \\
    \cline{3-8}
    \multicolumn{2}{|c|}{com PyramidORB + SIFT}
    & Cristo & Muralha & Panteão & Pirâmides & Taj Mahal & Torre Eiffel \\
    \hline
    \multirow{6}{*}{Classes Reais}
    & \multicolumn{1}{|c|}{Cristo} & 9 & 0 & 0 & 1 & 1 & 0 \\
    \cline{2-8}
    & \multicolumn{1}{|c|}{Muralha} & 9 & 0 & 0 & 1 & 0 & 0\\
    \cline{2-8}
    & \multicolumn{1}{|c|}{Panteão} & 0 & 0 & 8 & 0 & 0 & 0\\
    \cline{2-8}
    & \multicolumn{1}{|c|}{Pirâmides} & 3 & 0 & 0 & 6 & 1 & 0\\
    \cline{2-8}
    & \multicolumn{1}{|c|}{Taj Mahal} & 3 & 0 & 0 & 0 & 6 & 0\\
    \cline{2-8}
    & \multicolumn{1}{|c|}{Torre Eiffel} & 1 & 0 & 0 & 0 & 0 & 9\\
    \hline
  \end{tabular}


\end{document}